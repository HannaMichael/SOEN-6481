%%%%%%%%%%%%%%%%%%%%%%%%%%%%%%%%%%%%%%%%%
% Jacobs Landscape Poster
% LaTeX Template
% Version 1.0 (29/03/13)
%
% Created by:
% Computational Physics and Biophysics Group, Jacobs University
% https://teamwork.jacobs-university.de:8443/confluence/display/CoPandBiG/LaTeX+Poster
% 
% Further modified by:
% Nathaniel Johnston (nathaniel@njohnston.ca)
%
% This template has been downloaded from:
% http://www.LaTeXTemplates.com
%
% 
% Masaryk University presentation themes were downloaded from:
% https://www.overleaf.com/gallery/tagged/muni
%
% and ported into Jacobs Landscape Poster by:
% Jumaidil Awal (ideal1st.here@googlemail.com)
% 
% Jacobs Landscape Poster License:
% CC BY-NC-SA 3.0 (http://creativecommons.org/licenses/by-nc-sa/3.0/)
%
% Masaryk University's fibeamer theme license:
% Copyright 2015  Vít Novotný <witiko@mail.muni.cz>
% Faculty of Informatics, Masaryk University (Brno, Czech Republic)
% under Latex Project Public License
%
%%%%%%%%%%%%%%%%%%%%%%%%%%%%%%%%%%%%%%%%%

%----------------------------------------------------------------------------------------
%	PACKAGES AND OTHER DOCUMENT CONFIGURATIONS
%----------------------------------------------------------------------------------------

\documentclass[final]{beamer}

\usepackage[scale=1.24]{beamerposter} % Use the beamerposter package for laying out the poster

%\usetheme{confposter} % Use the confposter theme supplied with this template
\usetheme[faculty=chemo]{fibeamer} % Uncomment to use Masaryk University's fibeamer theme instead.

%\setbeamercolor{block title}{fg=ngreen,bg=white} % Colors of the block titles
%\setbeamercolor{block body}{fg=black,bg=white} % Colors of the body of blocks
%\setbeamercolor{block alerted title}{fg=white,bg=dblue!70} % Colors of the highlighted block titles
%\setbeamercolor{block alerted body}{fg=black,bg=dblue!10} % Colors of the body of highlighted blocks
% Many more colors are available for use in beamerthemeconfposter.sty

%-----------------------------------------------------------
% Define the column widths and overall poster size
% To set effective sepwid, onecolwid and twocolwid values, first choose how many columns you want and how much separation you want between columns
% In this template, the separation width chosen is 0.024 of the paper width and a 4-column layout
% onecolwid should therefore be (1-(# of columns+1)*sepwid)/# of columns e.g. (1-(4+1)*0.024)/4 = 0.22
% Set twocolwid to be (2*onecolwid)+sepwid = 0.464
% Set threecolwid to be (3*onecolwid)+2*sepwid = 0.708

\newlength{\sepwid}
\newlength{\onecolwid}
\newlength{\twocolwid}
\newlength{\threecolwid}
\setlength{\paperwidth}{46.8in} % A0 width: 46.8in
\setlength{\paperheight}{33.1in} % A0 height: 33.1in
\setlength{\sepwid}{0.024\paperwidth} % Separation width (white space) between columns
\setlength{\onecolwid}{0.21\paperwidth} % Width of one column
\setlength{\twocolwid}{0.451\paperwidth} % Width of two columns
\setlength{\threecolwid}{0.678\paperwidth} % Width of three columns
%\setlength{\topmargin}{-0.5in} % Reduce the top margin size
%-----------------------------------------------------------

\usepackage{graphicx}  % Required for including images

\usepackage{booktabs} % Top and bottom rules for tables

%----------------------------------------------------------------------------------------
%	TITLE SECTION 
%----------------------------------------------------------------------------------------

\title{The Plastic Number} % Poster title

\author{Michael Hanna} % Author(s)

\institute{Concordia University} % Institution(s)

%----------------------------------------------------------------------------------------


\usepackage{ragged2e}
 
\let\olditem=\item% 
\renewcommand{\item}{\olditem \justifying}%



\begin{document}
\addtobeamertemplate{block end}{}{\vspace*{2ex}} % White space under blocks
\addtobeamertemplate{block example end}{}{\vspace*{2ex}} % White space under example blocks
\addtobeamertemplate{block alerted end}{}{\vspace*{2ex}} % White space under highlighted (alert) blocks

\setlength{\belowcaptionskip}{2ex} % White space under figures
\setlength\belowdisplayshortskip{2ex} % White space under equations
%\begin{darkframes} % Uncomment for dark theme, don't forget to \end{darkframes}
\begin{frame} % The whole poster is enclosed in one beamer frame

%==========================Begin Head===============================

  \begin{columns}
   \begin{column}{\linewidth}
    \vskip1cm
    \centering
    \usebeamercolor{title in headline}{\color{fg}\Huge{\textbf{\inserttitle}}\\[0.5ex]}
    \usebeamercolor{author in headline}{\color{fg}\Large{\insertauthor}\\[1ex]}
    \usebeamercolor{institute in headline}{\color{fg}\large{\insertinstitute}\\[1ex]}
    \vskip1cm
   \end{column}
   \vspace{1cm}
  \end{columns}
 \vspace{1cm}

%==========================End Head===============================

\begin{columns}[t] % The whole poster consists of three major columns, the second of which is split into two columns twice - the [t] option aligns each column's content to the top

\begin{column}{\sepwid}\end{column} % Empty spacer column

\begin{column}{\onecolwid} % The first column

%----------------------------------------------------------------------------------------
%	OBJECTIVES
%----------------------------------------------------------------------------------------

\begin{exampleblock}{About the project}

The Plastic number is known as the Plastic constant or Plastic ratio.\newline
\newline The value of the Plastic number is\newline
\begin{eqnarray}
{\displaystyle \rho ={\sqrt[{3}]{\frac {9+{\sqrt {69}}}{18}}}+{\sqrt[{3}]{\frac {9-{\sqrt {69}}}{18}}}}
\end{eqnarray}
\newline Its decimal representation begin with 1.32471.

\end{exampleblock}

%----------------------------------------------------------------------------------------
%	INTRODUCTION
%----------------------------------------------------------------------------------------
\begin{exampleblock}{Challenges}
\justifying
\textbf{1 - Domain Knowledge}\newline\newline
Understanding the plastic number was challenging because this was the first time to be introduced to it. \newline

\begin{figure}
\includegraphics[width=1\linewidth]{img/understanding}\newline
\end{figure}



%------------------------------------------------


\textbf{2 - Interview}\newline\newline
\justifying
Finding a suitable interviewee that had experience in this specific domain was challenging undertaking considering the time constraints surrounding the project, preparing the questions and finally conducting the interview.
\end{exampleblock}


%------------------------------------------------



%----------------------------------------------------------------------------------------

\end{column} % End of the first column

\begin{column}{\sepwid}\end{column} % Empty spacer column

\begin{column}{\twocolwid} % Begin a column which is two columns wide (column 2)

\begin{columns}[t,totalwidth=\twocolwid] % Split up the two columns wide column

\begin{column}{\onecolwid}\vspace{-.74in} % The first column within column 2 (column 2.1)

%----------------------------------------------------------------------------------------
%	MATERIALS
%----------------------------------------------------------------------------------------
\textbf{3 - Finding an application}\newline
\justifying
Although the plastic number is well known, it was difficult to find an application to implement it, However after researching online, a suitable application for the constant was found for implementation. \newline

\textbf{4 - Implementation}\newline
\justifying

The implementation process was challenging as because one of the constraints of the project was not utilizing java's math library, therefore every function was needed to be implemented from scratch such as : Square root and cubic root.\newline
\begin{figure}
\includegraphics[width=1\linewidth]{img/programming3.jpg}\newline
\end{figure}

%----------------------------------------------------------------------------------------

\end{column} % End of column 2.1
\begin{column}{\sepwid}\end{column} % Empty spacer column

\begin{column}{\onecolwid}\vspace{-.74in} % The second column within column 2 (column 2.2)

%----------------------------------------------------------------------------------------
%	METHODS
%----------------------------------------------------------------------------------------

\textbf{5 - Time Constraints}\newline
\justifying
It was challenging to meet the deadlines in the project because of the length of this course was 6.5 weeks only.\newline

\newline
\begin{figure}
\includegraphics[width=1\linewidth,height=0.9\linewidth]{img/deadline1.jpeg}\newline
\end{figure}

\begin{exampleblock}{Decision Making}
\textbf{1 - Memento Design pattern}\newline

\begin{itemize}

\item It records the state of an object.
\item It eliminates the need to create multiple objects for saving purposes.
\item It has High Cohesion.
\end{itemize}
\end{exampleblock}

%----------------------------------------------------------------------------------------

\end{column} % End of column 2.2

\end{columns} % End of the split of column 2 - any content after this will now take up 2 columns width

%----------------------------------------------------------------------------------------
%	IMPORTANT RESULT
%----------------------------------------------------------------------------------------

\begin{alertblock}{Lesson Learned}
\justifying

\textbf{1. There will always be problems:} we shouldn't fear from the unknown domain because there is time to learn new things

\textbf{2. Time Management  :} This project allowed me to manage my time more effectively and efficiently allowing me to study and tackle the challenges of this project at the same time. 

\textbf{3. Applying what we learned  :} it was important to have a practical implementation of what we have studied in class.  

\end{alertblock} 

%----------------------------------------------------------------------------------------

\begin{columns}[t,totalwidth=\twocolwid] % Split up the two columns wide column again

\begin{column}{\onecolwid} % The first column within column 2 (column 2.1)

%----------------------------------------------------------------------------------------
%	MATHEMATICAL SECTION
%----------------------------------------------------------------------------------------



%----------------------------------------------------------------------------------------

\end{column} % End of column 2.1
\begin{column}{\sepwid}\end{column} % Empty spacer column

\begin{column}{\onecolwid} % The second column within column 2 (column 2.2)

%----------------------------------------------------------------------------------------
%	RESULTS
%----------------------------------------------------------------------------------------


%----------------------------------------------------------------------------------------

\end{column} % End of column 2.2

\end{columns} % End of the split of column 2

\end{column} % End of the second column

\begin{column}{\sepwid}\end{column} % Empty spacer column

\begin{column}{\onecolwid} % The third column

%----------------------------------------------------------------------------------------
%	CONCLUSION
%----------------------------------------------------------------------------------------



%----------------------------------------------------------------------------------------
%	ADDITIONAL INFORMATION
%----------------------------------------------------------------------------------------

\textbf{2 - Implementation of user stories}\newline
\justifying

Among the seven user stories that have been created, the following user stories have been selected:\newline
\begin{itemize}
\item \textbf{Basic Calculation:} A calculator can not be a calculator without general arithmetic calculations (Add, Subtract, Multiply \& Divide).\newline 
\item \textbf{Generate plastic number:} This is the core function that generate the plastic number to be utilized by the calculator.\newline 
\item \textbf{An Application of the number:} Implementing an application for the plastic number is important to understand the number itself.\newline 
\item \textbf{Enter two numbers:} All basic operation required to have at least two numbers to perform the calculation.\newline 
\item \textbf{Save results:} it was a good idea to implement this user story to use memento design pattern for better understanding of its utilization.\newline
\end{itemize}

\begin{figure}
\includegraphics[width=0.8\linewidth,height=0.5\linewidth]{img/userstory4.png}\newline
\end{figure}


%----------------------------------------------------------------------------------------
%	REFERENCES
%----------------------------------------------------------------------------------------


%----------------------------------------------------------------------------------------
%	ACKNOWLEDGEMENTS
%----------------------------------------------------------------------------------------

%\setbeamercolor{block title}{fg=red,bg=white} % Change the block title color

%\begin{exampleblock}{Acknowledgements}

%\small{\rmfamily{Nam mollis tristique neque eu luctus. Suspendisse rutrum congue nisi sed convallis. Aenean id neque dolor. Pellentesque habitant morbi tristique senectus et netus et malesuada fames ac turpis egestas.}} \\

%\end{exampleblock}

%----------------------------------------------------------------------------------------
%	CONTACT INFORMATION
%----------------------------------------------------------------------------------------

%\setbeamercolor{block alerted title}{fg=black,bg=norange} % Change the alert block title colors
%\setbeamercolor{block alerted body}{fg=black,bg=white} % Change the alert block body colors



%----------------------------------------------------------------------------------------

\end{column} % End of the third column

\begin{column}{\sepwid}\end{column} % Empty spacer column

\end{columns} % End of all the columns in the poster

\end{frame} % End of the enclosing frame
%\end{darkframes} % Uncomment for dark theme
\end{document}
