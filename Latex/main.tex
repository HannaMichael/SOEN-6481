\documentclass{article}
\usepackage[utf8]{inputenc}

\title{N11: Plastic Number}
\author{Michael Hanna - 40075977 }
\date{}

\usepackage{natbib}
\usepackage{graphicx}

\begin{document}

\maketitle

\section{Introduction}
The Plastic number is known as the Plastic constant or Plastic ratio.\newline
\newline The value of the Plastic number is
\begin{eqnarray}
{\displaystyle \rho ={\sqrt[{3}]{\frac {9+{\sqrt {69}}}{18}}}+{\sqrt[{3}]{\frac {9-{\sqrt {69}}}{18}}}}
\end{eqnarray}
\newline Its decimal representation begin with 1.32471.\citep{pn}
\section{Uses}
It can be a way of partitioning a square into three rectangles.\newline
\newline A Rectangle of aspect ratio ${\rho}^2$  can be used for dissections of a square into similar rectangles \citep{pn}


\begin{figure}[h!]
\centering
\includegraphics[scale=0.5]{square}
\caption{Three partitions of a square into similar rectangles}
\label{fig:squares}
\end{figure}


\bibliographystyle{plain}
\bibliography{references}
\end{document}
