\documentclass{article}
\usepackage[utf8]{inputenc}

\title{N6: Liouville Constant (c)}
\author{Michael Hanna - 40075977}
\date{}

\usepackage{natbib}
\usepackage{graphicx}

\begin{document}

\maketitle

\section{Introduction}
Liouville constant is also well known by Liouville number and it can be defined by:

\begin{equation}
c =\sum_{n=0}^{\infty} 10^{-n!} = \sum_{n=0}^{\infty}\frac{1}{10^{n!}}
\end{equation}
\begin{equation}
c =\frac{1}{10^1} + \frac{1}{10^2} +\frac{1}{10^6}+ \frac{1}{10^{24}} + ... 
\end{equation}

\begin{equation}
c = 0.110001000000000000000001000
\end{equation}

\section{Characteristics of Liouville Constant}
Liouville Constant was created in 1844 by Joseph Liouville \newline
\newline Liouville constant is the first transcendental number to be proven.\newline
\newline Transcendental number is a number that is not a root of any nonzero integer polynomial\newline
 \newline  Liouville constant is unique beacuse of its decimal fraction as it is a series of 1s and 0s. ones are in each decimal place corresponding to n!, and zeros everywhere else.\citep{lc}
\bibliographystyle{plain}
\bibliography{references}
\end{document}
