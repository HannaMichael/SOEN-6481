\documentclass{report}
\usepackage[utf8]{inputenc}

\title{N11: Plastic Number
\\ Problem 6 - 8}
\author{Michael Hanna - 40075977}
\date{}

\usepackage{natbib}
\usepackage{graphicx}
\usepackage{rotating}
\usepackage{tikz}

\def\checkmark{\tikz\fill[scale=0.4](0,.35) -- (.25,0) -- (1,0.7) -- (.25,.15) -- cycle;} 

\begin{document}
\maketitle
\tableofcontents
\listoftables
\chapter{User Story}
\section{User Story \# 1}
\begin{tabular}{|p{3cm}|p{9cm}| }
\hline
\multicolumn{2}{|c|}{US\#1 - Add two Numbers}\\
\hline
\textbf {Story \#}& 1\\
\hline
\textbf{Description}& As a user, I want to add two numbers, so that i can see the summation\\
\hline
\textbf{Acceptance Test}&And I know I am done when 2 + 3 = 5\\
\hline
\textbf{Estimated Points} & 1 Point\\
\hline
\textbf{Priority} & High \\
\hline
\textbf{Constrains}& The display should print up to 11 digits number. \\
\hline
\end{tabular}
\section{User Story \# 2}

\begin{tabular}{|p{3cm}|p{9cm}| }
\hline
\multicolumn{2}{|c|}{US\#2 - Subtract two Numbers} \\
\hline
\textbf {Story \#}& 2\\
\hline
\textbf{Description}& As a user, I want to subtract two numbers, so that i can see the difference\\
\hline
\textbf{Acceptance Test}&And I know I am done when 3 - 2 = 1\\
\hline
\textbf{Estimated Points} & 1 Point \\
\hline
\textbf{Priority} & High \\
\hline
\textbf{Constrains}& The display should print up to 11 digits number. \\
\hline
\end{tabular}

\section{User Story \# 3}
\begin{tabular}{|p{3cm}|p{9cm}| }
\hline
\multicolumn{2}{|c|}{US\#3 - Multiply two Numbers} \\
\hline
\textbf {Story \#}& 3\\
\hline
\textbf{Description}& As a user, I want to multiply two numbers, so that i can see the product\\
\hline
\textbf{Acceptance Test}& And I know I am done when 5 x 2 = 10\\
\hline
\textbf{Estimated Points} & 1 Point \\
\hline
\textbf{Priority} & High \\
\hline
\textbf{Constrains}& The display should print up to 11 digits number. \\
\hline
\end{tabular}

\section{User Story \# 4}
\begin{tabular}{|p{3cm}|p{9cm}| }
\hline
\multicolumn{2}{|c|}{US\#4 - Divide two Numbers} \\
\hline
\textbf {Story \#}& 4\\
\hline
\textbf{Description}& As a user, I want to multiply two numbers, so that i can see the quotient\\
\hline
\textbf{Acceptance Test}& And I know I am done when 6 / 2 = 3\\
\hline
\textbf{Estimated Points} & 1 Point \\
\hline
\textbf{Priority} & High \\
\hline
\textbf{Constrains}& The display should print up to 11 digits number. also, the second number shall not be zero.\\
\hline
\end{tabular}

\section{User Story \# 5}
\begin{tabular}{|p{3cm}|p{9cm}| }
\hline
\multicolumn{2}{|c|}{US\#5 - Clear the wrong digit} \\
\hline
\textbf {Story \#}& 5\\
\hline
\textbf{Description}& As a user, I want to clear the wrong digit, so that i can update the number without entering the whole number again\\
\hline
\textbf{Acceptance Test}& And I know I am done when 123455 can be changed to 123456 by changing the last digit \\
\hline
\textbf{Estimated Points} & 1 Point \\
\hline
\textbf{Priority} & Medium \\
\hline
\textbf{Constrains}& The display should print up to 11 digits number.\\
\hline
\end{tabular}

\section{User Story \# 6}
\begin{tabular}{|p{3cm}|p{9cm}| }
\hline
\multicolumn{2}{|c|}{US\#6 - Save number in Memory} \\
\hline
\textbf {Story \#}& 6\\
\hline
\textbf{Description}& As a user, I want to save the number in memory, so that i can use it later.\\
\hline
\textbf{Acceptance Test}& And I know I am done when I press M  the number should be stored in memory and letter M should be shown in the bar. \\
\hline
\textbf{Estimated Points} & 2 Points \\
\hline
\textbf{Priority} & Medium \\
\hline
\textbf{Constrains}& It is limited to the size of the actual memory.\\
\hline
\end{tabular}

\section{User Story \# 7}
\begin{tabular}{|p{3cm}|p{9cm}| }
\hline
\multicolumn{2}{|c|}{US\#7 - Save operations even if i pressed clear by mistake} \\
\hline
\textbf {Story \#}& 7\\
\hline
\textbf{Description}& As a user, I want to save the equation in memory, so that i can recall it again even if I pressed clear by mistake.\\
\hline
\textbf{Acceptance Test}& And I know I am done when I press recall button the previous equation should be recalled from memory. \\
\hline
\textbf{Estimated Points} & 3 Points \\
\hline
\textbf{Priority} & High \\
\hline
\textbf{Constrains}& Huge usage of memory to store everything that has been entered into the calculator.\\
\hline
\end{tabular}

\section{User Story \# 8}
\begin{tabular}{|p{3cm}|p{9cm}| }
\hline
\multicolumn{2}{|c|}{US\#8 - Get Plastic number} \\
\hline
\textbf {Story \#}& 8\\
\hline
\textbf{Description}& As a user, I want to get the plastic number, so that i can perform some operations.\\
\hline
\textbf{Acceptance Test}& And I know I am done when I press plastic number button, I get 1.324717957. \\
\hline
\textbf{Estimated Points} & 3 Points \\
\hline
\textbf{Priority} & High \\
\hline
\textbf{Constrains}& The display should print 11 digits number.\\
\hline
\end{tabular}

\section{User Story \# 9}
\begin{tabular}{|p{3cm}|p{9cm}| }
\hline
\multicolumn{2}{|c|}{US\#9 - Calculate circumradius of Snub Icosidodecadodecahedron} \\
\hline
\textbf {Story \#}& 9\\
\hline
\textbf{Description}& As a user, I want to calculate circumradius of Snub Icosidodecadodecahedron.\\
\hline
\textbf{Acceptance Test}& And I know I am done when I get the result 1.12689 \\
\hline
\textbf{Estimated Points} & 5 Points \\
\hline
\textbf{Priority} & Low \\
\hline
\textbf{Constrains}& The usage of this function is only valid for a = 1.\\
\hline
\end{tabular}

\section{User Story \# 10}
\begin{tabular}{|p{3cm}|p{9cm}| }
\hline
\multicolumn{2}{|c|}{US\#11 - Enter two numbers} \\
\hline
\textbf {Story \#}& 11\\
\hline
\textbf{Description}& As a user, I want to enter two numbers so that i can perform some operations.\\
\hline
\textbf{Acceptance Test}& And I know I am done when I pressed 123 i got 123 on the screen. \\
\hline
\textbf{Estimated Points} & 1 Point \\
\hline
\textbf{Priority} & High \\
\hline
\textbf{Constrains}& The display should print 11 digital numbers.\\
\hline
\end{tabular}


\chapter{Backward Traceability Matrix}

\begin{table*}[!ht]
\centering
\addtolength{\leftskip} {-2cm}
\addtolength{\rightskip}{-2cm}

\begin{tabular}{|p{2cm}|p{2cm}|p{2cm}|p{2cm}|p{2cm}|}

\hline
& Interviewee& Internet & Life Experience & Project Description\\
\hline

US\#1&&&\checkmark&\\
\hline
US\#2 &&&\checkmark&\\
\hline
US\#3&&&\checkmark&\\
\hline

US\#4&&&\checkmark&\\
\hline

US\#5&&&\checkmark&\\
\hline

US\#6&&&\checkmark&\\
\hline

US\#7&\checkmark&&&\\
\hline

US\#8&&&&\checkmark\\
\hline

US\#9&&\checkmark&&\\
\hline

US\#10&&&\checkmark&\\
\hline

\end{tabular}
\caption{Backward Traceability Matrix}

\end{table*}
\end{document}
