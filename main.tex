\documentclass{report}
\usepackage[utf8]{inputenc}

\title{N11: Plastic Number
\\ Problem 6 - 8}
\author{Michael Hanna - 40075977}
\date{}

\usepackage{natbib}
\usepackage{graphicx}
\usepackage{rotating}
\usepackage{tikz}

\def\checkmark{\tikz\fill[scale=0.4](0,.35) -- (.25,0) -- (1,0.7) -- (.25,.15) -- cycle;} 

\begin{document}
\maketitle
\tableofcontents
\listoftables
\chapter{Repository Details}
The link for the repository is  \newline \url{https://github.com/HannaMichael/SOEN-6481}
\chapter{Changes from D1 to D2}
The Application of the plastic number has been changed to calculate the circumradius of Snub Icosidodecadodecahedron.
\chapter{User Story}
\section{User Story \# 1}
\begin{tabular}{|p{3.5cm}|p{9cm}| }
\hline
\multicolumn{2}{|c|}{Basic Arithmetic Calculation}\\
\hline
\textbf {Story ID}& US1\\
\hline
\textbf{Description}& As a user, I want to perform basic arithmetic calculation such as Add, Subtract, Multiply and Divide, so that i can get the summation, Difference, product and quotient\\
\hline
\textbf{Priority} & High \\
\hline
\textbf{Estimated Points} & 2 Points\\
\hline
\textbf{Constraints}& User shall perform the calculation on two numbers only with one operation, User shall not be able to perform compound operations.\\
\hline
\textbf{Acceptance Criteria}&
 -  User shall first select the desired operation [select a number from 1 to 4]\newline
 -  User shall enter the first number\newline
 -  User shall enter the second number\newline
 -  User shall press the enter button to do the calculation\\
\hline
\textbf{Acceptance Test}&And I know I am done when I perform the following calculations: \newline  2 + 3 = 5 \newline  3 - 2 = 1 \newline 5 x 2 = 10 \newline 6 / 2 = 3\\
\hline
\end{tabular}

\section{User Story \# 2}
\begin{tabular}{|p{3.5cm}|p{9cm}| }
\hline
\multicolumn{2}{|c|}{Save result in Memory} \\
\hline
\textbf {Story ID}& US2\\
\hline
\textbf{Description}& As a user, I want to save the number in memory, so that i can recall it later.\\
\hline
\textbf{Priority} & Medium \\
\hline
\textbf{Estimated Points} & 5 Points \\
\hline
\textbf{Constraints}& One Result only can be saved in the memory.\\
\hline
\textbf{Acceptance Criteria}& 

-   User shall first do any arithmetic calculation from user story ID ("US1") \newline
-   User shall get a message either to save the number in memory or not.\newline
-   If the user wants to save the number, then the user shall press "Y".\newline
-   If the user does not want to save the number, then the user shall press "N".\\
\hline
\textbf{Acceptance Test}& And I know I am done when I press "Y" the result shall be stored in memory and then i can select number "7" from the main menu to get the saved number.\\
\hline
\end{tabular}

\section{User Story \# 3}
\begin{tabular}{|p{3.5cm}|p{9cm}| }
\hline
\multicolumn{2}{|c|}{Clear the wrong digit} \\
\hline
\textbf {Story Id}& US3\\
\hline
\textbf{Description}& As a user, I want to clear the wrong digit, so that i can update the number without entering the whole number again\\
\hline
\textbf{Priority} & Low \\
\hline
\textbf{Estimated Points} & 1 Point \\
\hline
\textbf{Constraints}& N/A.\\
\hline
\textbf{Acceptance Criteria}&

-   User shall enter the first number.\newline
-   User shall press "E" to edit the number that he has entered.\\
\hline
\textbf{Acceptance Test}& And I know I am done when the user enter 123455 and the user can change the number to 123456 by changing the last digit only.\\
\hline
\end{tabular}

\section{User Story \# 4}
\begin{tabular}{|p{3.5cm}|p{9cm}| }
\hline
\multicolumn{2}{|c|}{Save operations even if i pressed clear by mistake} \\
\hline
\textbf {Story Id}& US4\\
\hline
\textbf{Description}& As a user, I want to save the equation in memory, so that i can recall it again even if I pressed clear by mistake.\\
\hline
\textbf{Priority} & Low \\
\hline
\textbf{Estimated Points} & 5 Points\\
\hline
\textbf{Constraints}& Huge usage of memory to store everything that has been entered into the calculator.\\
\hline
\textbf{Acceptance Criteria}& 
-   User shall enter the equation.\newline
-   User shall press clear to earse the equation.\newline
-   User shall be able to recall the equation again by pressing recall button.\\
\hline
\textbf{Acceptance Test}& And I know I am done when I press equation recall button and the previous equation should be recalled from memory. \\
\hline
\end{tabular}

\section{User Story \# 5}
\begin{tabular}{|p{3.5cm}|p{9cm}| }
\hline
\multicolumn{2}{|c|}{Get Plastic number} \\
\hline
\textbf {Story Id}& US5\\
\hline
\textbf{Description}& As a user, I want to get the plastic number, so that i can perform some operations.\\
\hline
\textbf{Priority} & High \\
\hline
\textbf{Estimated Points} & 3 Points \\
\hline
\textbf{Constraints}& The User shall not use the plastic number with the basic operations, the user can not use the plastic number as a first or second number in the basic operations.\\
\hline
\textbf{Acceptance Criteria}& 
-   User shall press number "5".\newline
-   User shall be able to see the plastic number.\\
\hline
\textbf{Acceptance Test}& And I know I am done when I press "5" Get plastic number and I get 1.324717957.\\
\hline
\end{tabular}

\section{User Story \# 6}
\begin{tabular}{|p{3.5cm}|p{9cm}| }
\hline
\multicolumn{2}{|c|}{Calculate circumradius of Snub Icosidodecadodecahedron\cite{pn}} \\
\hline
\textbf {Story Id}& US6\\
\hline
\textbf{Description}& As a user, I want to calculate circumradius of Snub Icosidodecadodecahedron.\\
\hline
\textbf{Priority} & High\\
\hline
\textbf{Estimated Points} & 5 Points \\
\hline
\textbf{Constraints}& The usage of this function is only valid for (Cartesian coordinate) a = 1.\\
\hline
\textbf{Acceptance Criteria}& 
-   User shall press number "6".\newline
-   User shall be able to see the circumradius of Snub Icosidodecadodecahedron.\\
\hline
\textbf{Acceptance Test}& And I know I am done when I press number "6" and I get the result 1.12689 \\
\hline
\end{tabular}

\section{User Story \# 7}
\begin{tabular}{|p{3.5cm}|p{9cm}| }
\hline
\multicolumn{2}{|c|}{Enter two numbers} \\
\hline
\textbf {Story Id}& US7\\
\hline
\textbf{Description}& As a user, I want to enter two numbers so that i can perform some operations on these numbers.\\
\hline
\textbf{Priority} & High \\
\hline
\textbf{Estimated Points} & 1 Point \\
\hline
\textbf{Constraints}& The range of each number shall be between $3.40282347x10^{38} , 1.40239846 x 10^{-45}$.\\
\hline
\textbf{Acceptance Criteria}&
-   User shall first select the desired operation [select a number from 1 to 4].\newline
-   User shall enter the first number and see the first number on the screen.\newline
-   User shall enter the second number and see the second number on the screen.\\
\hline
\textbf{Acceptance Test}& And I know I am done when I pressed 123 i got 123 on the screen then i pressed 456 and i got 456 on the screen.\\
\hline
\end{tabular}

\chapter{Backward Traceability Matrix}

\begin{table*}[!ht]
\centering
\addtolength{\leftskip} {-2cm}
\addtolength{\rightskip}{-2cm}

\begin{tabular}{|p{2cm}|p{2cm}|p{2cm}|p{2cm}|p{2cm}|}

\hline
& Interview & Persona & Internet  & Project Description\\
\hline

US1&\checkmark&\checkmark&&\\
\hline
US2 &\checkmark&\checkmark&&\\
\hline
US3&\checkmark&\checkmark&&\\
\hline
US4&\checkmark&\checkmark&&\\
\hline
US5&&&&\checkmark\\
\hline
US6&&&\checkmark&\\
\hline
US7&\checkmark&\checkmark&&\\
\hline

\end{tabular}
\caption{Backward Traceability Matrix}
\end{table*}
\newline\noindent The source of US1, US2, US3, US4 \& US7 is P.Tran Tuan.\\
\newline The source of US6 is \newline \url{http://mathworld.wolfram.com/SnubIcosidodecadodecahedron.html}
\newline
\chapter{Implementation}
The User stories that has been implemented are as follow:
\begin{itemize}
    \item US1 - Basic Arithmetic Calculation
    \item US2 - Save result in Memory
    \item US5 - Get Plastic Number
    \item US6 - Calculate circumradius of Snub Icosidodecadodecahedron
    \item US7 - Enter two numbers\newline
\end{itemize}
\newline Memento Design Pattern has been used\footnote{\url{https://www.tutorialspoint.com/design_pattern/memento_pattern}}.\newline
\newline Junit has been used for unit testing.
\bibliographystyle{plain}
\bibliography{references}

\end{document}
